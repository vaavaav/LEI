\documentclass[12pt, a4paper, titlepage, oneside, dvipsnames]{report}

%-----------------------------------------------------------
%	PACKAGES
%-----------------------------------------------------------
\usepackage[utf8]{inputenc} 
\usepackage[T1]{fontenc}  %accented characters,...
\usepackage[portuguese]{babel} 
\usepackage[dvipsnames]{xcolor}
\usepackage{amsmath}
\usepackage{textcomp}
\usepackage{amsfonts}
\usepackage{tikz}
\usepackage{tgbonum}
\usepackage{mathtools}

%-----------------------------------------------------------
%	SETTINGS
%-----------------------------------------------------------0

\DeclareMathSizes{12}{15}{15}{15} % text size, display/text style, script style, scriptscript style


%-----------------------------------------------------------
%	
%-----------------------------------------------------------0

\begin{document}


    \begin{titlepage}
       \begin{center}
           \vspace*{1cm}

           \textbf{Métodos Numéricos e Otimização Não Linear}

           \vspace{0.5cm}
            %qualquer cena

           \vspace{1.5cm}

           \textbf{Pedro Peixoto}

           \vfill

           Apontamentos e resumos\\

           \vspace{0.8cm}


           Unidade Curricular : Métodos Numéricos e Otimização Não Linear\\
           3º ano - Mestrado Integrado em Engenharia Informática (MIEI)\\
           Universidade do Minho\\ 
           Ano Letivo 2020/2021
           Portugal\\

       \end{center}
    \end{titlepage}

    \pagestyle{empty}


    \chapter{Métodos Numéricos}

    \section{Tipos de Erros}

    \subsection{\color{RoyalPurple} Incerteza nos dados}

        \begin{paragraph}\\
        Os dados de entrada contêm um {\color{Mahogany} imprecisão} inerent, isto é, não há como os evitar,
         uma vez que representam medidas obtidas usando equipamentos específicos (balanças, voltímetros, amperímetros, termómetros, ...).
        \end{paragraph}


        \begin{paragraph}\\
        Não existe nenhum equipamento de medição com uma precisão 100\%.
        \end{paragraph}

        \begin{paragraph}\\
        Como se poderá pesar átomos com uma balança de supermercado?
        \end{paragraph}

    \subsection{\color{RoyalPurple} Erro de arredondamento}

        \begin{paragraph}\\
        Surge na representação dos dados no computador - o tamanho da palavra no computador/calculadora tem {\color{Mahogany} tamanho limitado}.
        \end{paragraph}


        \begin{paragraph}\\
        Como se poderá representar sem erro o valor {\color{RoyalPurple} \(\frac{1}{3} = 0.(3)\)}?
        \end{paragraph}

        \begin{paragraph}\\
        E o valor de {\color{RoyalPurple} $\pi$} ? {\color{RoyalPurple} 3.14} ou {\color{RoyalPurple} 3.14159265358979} ???
        \end{paragraph}

        \begin{paragraph}\\
        Os erros de arrendondamento vão-se acumulando à medida que se vão realizando operações aritméticas e, a sua influência 
        no resultado pode ser muito ampliada. Muitas vezes, o resultado final de um conjunto de operações é diferente do resultado
        exato.
        \end{paragraph}

    \subsection{\color{RoyalPurple} Erro de truncatura}

        \begin{paragraph}\\
        Na substituição de um problema contínuo por um {\color{Mahogany} discreto} (diferenciação numérica e integração numérica).
        \end{paragraph}


        \begin{paragraph}\\
        Na substituição de um processo de cálculo infinito por um finito ({\color{Mahogany} métodos iterativos}).
        \end{paragraph}

        \begin{paragraph}{\color{OliveGreen} Exemplo}:\\
        Pretende-se calcular {\color{RoyalPurple} $e^x$} usando a expressão em série de Taylor da função:\\
        \[
            \color{RoyalPurple}
            e^x = 1 + x + \frac{1}{2!}x^2 + \frac{1}{3!}x^3 + ... + \frac{1}{n!}x^n + ...  
        \]\\
        Se se usar apenas os {\color{RoyalPurple} n} termos da série (se a "truncarmos" no {\color{RoyalPurple} enésimo} termo)
         obtém-se uma aproximação a {\color{RoyalPurple} $e^x$}:

        \[
            \color{RoyalPurple}
            e^x \approx 1 + x + \frac{1}{2!}x^2 + \frac{1}{3!}x^3 + ... + \frac{1}{(n-1)!}x^{n-1}  
        \]\\
        \end{paragraph}

    \subsection{\color{RoyalPurple} Formato Vírgula Flutuante}

        \[
            \color{RoyalPurple}
            fl(x) = \pm 0.d_1d_2...d_k \times b^e
        \]

        \begin{paragraph}\\
        A mantissa {\color{RoyalPurple} \(d_1d_2...d_k\)} é um número finito de digítos que define o comprimento da palavra ({\color{RoyalPurple} k})
         - a precisão aumenta com o aumento de {\color{RoyalPurple} k}.
        \end{paragraph}

        \begin{paragraph}\\
        {\color{RoyalPurple} \(b\)} é a base de representação.
        \end{paragraph}
        \begin{paragraph}\\
        {\color{RoyalPurple} \(e\)} é a expoente.
        \end{paragraph}

        \begin{paragraph}\\
        Se {\color{RoyalPurple} \(d_1 \neq 0\)} o formato diz-se {\color{Mahogany} normalizado}, por exemplo, 
        \[
            \color{RoyalPurple}
            fl(x) = \pm 0.d0045 \times 10^0, fl_{norm}(x) = 0.45 \times 10^{-2} 
        \].
        \end{paragraph}

    \subsection{\color{RoyalPurple} Erro Absoluto}
        Considere-se:
        \(\color{RoyalPurple} \overline{x} \in I\!R\) - o valor exato de um número.\\
        \(\color{RoyalPurple} x \in I\!R\) - o valor aproximado (o que vai ser utilizado nos cálculos).\\

        \begin{paragraph}\\ 
        \(\color{RoyalPurple} d_x = \lvert \overline{x} - x \rvert\) - o {\color{Mahogany} erro absoluto} (desconhecido).
        \end{paragraph}
        \begin{paragraph}\\
        \(\color{RoyalPurple} \lvert \overline{x} - x \rvert \leq \delta_x\) - o limite superior do erro absoluto.
        \end{paragraph}
        \begin{paragraph}\\
        \(\color{RoyalPurple} \overline{x} \in [ \, x - \delta_x, x + \delta_x ] \,\) - o intervalo de incerteza.
        \end{paragraph}

    \subsection{\color{RoyalPurple} Erro relativo}

        \begin{paragraph}\\ 
        \(\color{RoyalPurple} r_x = \frac{\lvert \overline{x} - x \rvert}{\lvert \overline{x} \rvert}, \overline{x} \neq 0 \),
         o {\color{Mahogany} erro relativo} (desconhecido).
        \end{paragraph}
        \begin{paragraph}\\
        \(\color{RoyalPurple} r_x \approx \frac{\lvert \overline{x} - x \rvert}{\lvert \overline{x} \rvert} \leq \frac{\delta_x}{\lvert x \rvert}\) 
        - {\color{RoyalPurple} 100\% $r_x$} - percentagem do erro.
        \end{paragraph}
        \begin{paragraph}\\
        {\color{Mahogany} Nota:} o erro absoluto não permite avaliar a importância do erro cometido.
        \end{paragraph}
    
    \subsection{\color{RoyalPurple} Majorante do erro}
        Associado a um número $\color{RoyalPurple} x$ está o limite superior do erro absoluto $\color{RoyalPurple} \delta_x$:\\
        - $\color{RoyalPurple} \delta_x$ é metade da última casa decimal de $\color{RoyalPurple} x$.\\
        Por exemplo, se $\color{RoyalPurple} \delta_x$ for $\color{RoyalPurple} 15.23$ (a última casa decimal é a das centésimas), logo:
        \[\color{RoyalPurple}
        \delta_x = 0.5 \times 0.01 = 0.005, \overline{x} \in [ \, 15.23 - 0.005, 15.23 + 0.005 ] \,
        \]

    \subsection{\color{RoyalPurple} Fórmula Fundamental do Erro}
    {\color{RoyalPurple} \underline{Teorema}}
    \\Seja $\color{RoyalPurple} \overline{x} \in I_x = [ \, x - \delta_x, x + \delta_x ] \,$ e \\
         $\color{RoyalPurple} \overline{y} \in I_y = [ \, y - \delta_y, y + \delta_y ] \,$\\
    ($\color{RoyalPurple} x$ e $\color{RoyalPurple} y$ representam valores aproximados dos valores exatos $\color{RoyalPurple} x$ e $\color{RoyalPurple} y$)



 




\end{document}